\documentclass{ujarticle}
\usepackage[dvipdfmx]{graphicx}
\usepackage{listings}
\lstset{%
  language={Python},
  breaklines=true
}
\usepackage{geometry}
\geometry{left=25mm,right=25mm,top=30mm,bottom=30mm}

\title {課題1レポート}

\author{芦田聖太}

\date{提出日 17/12/21}

\begin{document}


\begin{titlepage}
\maketitle
\thispagestyle{empty}
\end{titlepage}


\section*{課題1}
\section{3層のニューラルネットワークの構築}
MNIST の画像 1 枚を入力とし,3 層ニューラルネットワークを用いて,0~9 の値のう
ち 1 つを出力するプログラムを作成する。

\subsection*{仕様}
\begin{itemize}
\item キーボードから 0~9999 の整数を入力 i として受け取り,0~9 の整数を標準出力に
出力すること。
\item MNIST のテストデータ 10000 枚の画像のうち i 番目の画像を入力画像として用いる。
(ただし、MNIST の画像サイズ(28 × 28),画像枚数(10000 枚),クラス数(C = 10)は既
知とする。)
\item 中間層のノード数 M は自由に決めて良い。
\item 重み W(1), W(2), b(1), b(2) については乱数で決定すること。ここでは,手前の層の
ノード数を N として 1/N を分散とする平均 0 の正規分布で与えることとする。実行する度に同じ結果を出力するよう乱数のシードを固定すること 。
\end{itemize}


\section{設計方針}
仕様を満たす3層のニューラルネットワークを構築するのに必要なものを以下にあげる。これらを設計し、組み合わせることで3層のニューラルネットワークを構成する。
\begin{itemize}
\item キーボードからの入力処理と画像の取り込み
\item 中間層への入力と出力層の入力の計算
\item シグモイド関数
\item ソフトマックス関数
\end{itemize}

\section{実装とプログラムの説明}
\subsection{キーボードからの入力処理と画像の取り込み}
X, Yにテストデータを取り込み、Xは(10000, 28, 28)の3次元の配列に整える。Yは正解のラベルの集合なので長さ10000の配列とする。\\
キーボードの入力に関しては、whileループの中で入力をstrnumに格納し、数字が0~9999の場合はbreakする。それ以外の場合はループして0~9999の数字が入力されるまで待つ。break後、Xの入力された番号の画像データ(28 , 28)をindataに格納しさらに(784, 1)の配列へと変換する。\\

main.py
\begin{lstlisting}[basicstyle=\ttfamily\footnotesize, frame=single]
# 入力
# 画像取り込み
mndata = MNIST("/Users/omushota/ex4-image/le4nn")
X, Y = mndata.load_testing()
X = np.array(X)
X = X.reshape((X.shape[0], 28, 28))
Y = np.array(Y)


# キーボード入力待ち
while True:
    strnum = input("input number : ")
    num = int(strnum)
    if (num < 0) or (num > 9999):
        print("Please type 0 ~ 9999")
    else:
        break

indata = X[num]
line = X.shape[0]
row = X.shape[1]
indata = np.reshape(indata, (row * row, 1))
\end{lstlisting}


\subsection{中間層への入力と出力層への入力}
中間層への入力と出力層への入力に関しては同じ処理を行うので、中間層への入力についてのみについて述べる。課題2で同じ処理を複数のデータについて行うことを考慮して、中間層への入力を計算する関数を構成した。関数では、入力データ, 中間層の数, 入力層の数, 平均, 分散, シード値を入力とする。まず、np.random.seed(seed)で乱数のシード値を設定することで、実行するたびに同じ乱数が生成されるようにする。次に、row * middleの長さの乱数配列を発生させ、(middle, row)の2次元配列にしweightに格納する。また、同様にmiddleの長さの乱数配列を発生させ、(middle, 1)の2次元配列にしbに格納する。最後に、weightと入力データの積にbを足したものを返す。この関数をmain.pyで用いて中間層への入力とした。\\

layer.py
\begin{lstlisting}[basicstyle=\ttfamily\footnotesize, frame=single]
import numpy as np


def mid(indata, middle, row, average, variance, seed):
    np.random.seed(seed)
    weight = np.random.normal(average, variance, row * middle)
    weight = np.reshape(weight, (middle, row))
    b = np.random.normal(average, variance, middle)
    b = np.reshape(b, (middle, 1))
    return weight.dot(indata) + b

def endend(midout, end, middle, average, variance, seed):
    np.random.seed(seed)
    weight1 = np.random.normal(0, variance, middle * end)
    weight1 = np.reshape(weight1, (end, middle))
    b1 = np.random.normal(average, variance, end)
    b1 = np.reshape(b1, (end, 1))
    return weight1.dot(midout) + b1
\end{lstlisting}

main.py
\begin{lstlisting}[basicstyle=\ttfamily\footnotesize, frame=single]
# 中間層
middle = 4
average = 0
variance = math.sqrt(1/line)
seed = 100
midinput = mid(indata, middle, row * row, average, variance, seed)

~~~~~~

# 出力層
end = 10
average1 = 0
variance1 = math.sqrt(1/middle)
fininput = endend(midout, end, middle, average1, variance1, seed)
\end{lstlisting}


\subsection{シグモイド関数}
基本的に返り値は計算式のままであるが、入力によって少しだけ出力を変えている。オーバーフローの処理のために、入力が34.538776394910684より大きいときは1.0 - 1e-15を出力する。また、入力が-34.538776394910684より小さいときは1e-15を出力する。そしてそれ以外の時は、1.0 / (1.0 + np.exp(-x))を出力するようにした。\\

sigmoid.py
\begin{lstlisting}[basicstyle=\ttfamily\footnotesize, frame=single]
import numpy as np

@np.vectorize

def sigmoid(x):
    sigmoid_range = 34.538776394910684
    if x <= -sigmoid_range:
        return 1e-15
    if x >= sigmoid_range:
        return 1.0 - 1e-15
    return 1.0 / (1.0 + np.exp(-x))
\end{lstlisting}

\subsection{ソフトマックス関数}
cにnp.maxで配列中の1番大きな値を格納する。a-cでa中の各要素からそれぞれcを引いた配列を作り、np.exp(a-c)で各要素がネイピア数の指数となった配列をexp\_aに格納する。最後にexp\_aの総和を計算しsum\_exp\_aとしてexp\_a / sum\_exp\_aを返す。

softmax.py
\begin{lstlisting}[basicstyle=\ttfamily\footnotesize, frame=single]
import numpy as np


def softmax(a):
    # 一番大きい値を取得
    c = np.max(a)
    # 各要素から一番大きな値を引く(オーバーフロー対策)
    exp_a = np.exp(a - c)
    sum_exp_a = np.sum(exp_a)
    # 要素の値/全体の要素の合計
    y = exp_a / sum_exp_a

    return y
\end{lstlisting}
    
\subsection{全体}
今まで説明したものを合わせたものを、以下に載せる。最後の部分が未説明なのでその部分だけ説明を加える。最後の部分ではソフトマックスの出力の配列の最大値のインデックスを取り出し、出力している。

main.py
\begin{lstlisting}[basicstyle=\ttfamily\footnotesize, frame=single]
import sigmoid
import softmax
import  CrossEntropy
import math
import numpy as np
from mnist import MNIST
from layer import mid, endend


# 入力層
# 画像取り込み
mndata = MNIST("/Users/omushota/ex4-image/le4nn")
X, Y = mndata.load_testing()
X = np.array(X)
X = X.reshape((X.shape[0], 28, 28))
Y = np.array(Y)


# キーボード入力待ち
while True:
    strnum = input("input number : ")
    num = int(strnum)
    if (num < 0) or (num > 9999):
        print("Please type 0 ~ 9999")
    else:
        # print("break")
        break

indata = X[num]
line = X.shape[0]
row = X.shape[1]
indata = np.reshape(indata, (row * row, 1))


# 中間層
middle = 4
average = 0
variance = math.sqrt(1/line)
seed = 100
midinput = mid(indata, middle, row, average, variance, seed)

# シグモイド
midout = sigmoid.sigmoid(midinput)


# 出力層
end = 10
average1 = 0
variance1 = math.sqrt(1/middle)
fininput = endend(midout, end, middle, average1, variance1, seed)

# ソフトマックス
finout = softmax.softmax(fininput)
indexmax = np.where(finout == finout.max())[0][0]
print(indexmax)
\end{lstlisting}

\section{実行結果}
以下に2つの実行結果を示す。inputの値が変わることで出力結果も変わることがわかる。また、0~9999の数字でしか次に進まないことが確認できた。
\begin{lstlisting}[basicstyle=\ttfamily\footnotesize, frame=single]
input number : 10
6

Process finished with exit code 0
\end{lstlisting}

\begin{lstlisting}[basicstyle=\ttfamily\footnotesize, frame=single]
input number : 999999999
Please type 0 ~ 9999
input number : -13
Please type 0 ~ 9999
input number : 12
4

Process finished with exit code 0
\end{lstlisting}

\section{考察}
\subsection{工夫点}
入力に関しては、0~9999までの数字が入らない限り待ち続けるようにした部分が工夫した点と言える。また、シグモイド関数に関してもオーバーフロー対策として入力によって値を変えた部分が自分なりに工夫した部分である。さらに、中間層への入力と出力層への入力を関数を生成して計算した部分も工夫した点である。そうすることで、課題2のミニバッチ処理がスムーズになることが予想される。

\subsection{問題点}
課題3の逆伝播の部分で重みを更新していかなければならないが、この仕様のままだと重みが関数に隠れているので、重みの更新が今の所問題点として考えられる。

\end{document}
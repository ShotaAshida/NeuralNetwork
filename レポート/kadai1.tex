\documentclass{ujarticle}
\usepackage[dvipdfmx]{graphicx}
\usepackage{listings}
\lstset{%
  language={Java},
  breaklines=true
}
\usepackage{geometry}
\geometry{left=25mm,right=25mm,top=30mm,bottom=30mm}

\title {課題1レポート}
\author{芦田聖太}
\date{提出日 17/12/21}

\begin{document}


\begin{titlepage}
\maketitle
\thispagestyle{empty}
\end{titlepage}


\section*{課題1}
\section*{3層のニューラルネットワークの構築}
MNIST の画像 1 枚を入力とし,3 層ニューラルネットワークを用いて,0~9 の値のう
ち 1 つを出力するプログラムを作成する。

\subsection*{仕様}
\begin{itemize}
\item キーボードから 0~9999 の整数を入力 i として受け取り,0~9 の整数を標準出力に
出力すること。
\item MNIST のテストデータ 10000 枚の画像のうち i 番目の画像を入力画像として用いる。
(ただし、MNIST の画像サイズ(28 × 28),画像枚数(10000 枚),クラス数(C = 10)は既
知とする。)
\item 中間層のノード数 M は自由に決めて良い。
\item 重み W(1), W(2), b(1), b(2) については乱数で決定すること。ここでは,手前の層の
ノード数を N として 1/N を分散とする平均 0 の正規分布で与えることとする。実行する度に同じ結果を出力するよう乱数のシードを固定すること 。
\end{itemize}


\section*{設計方針}



\begin{lstlisting}[basicstyle=\ttfamily\footnotesize, frame=single]
\end{lstlisting}

\begin{lstlisting}[basicstyle=\ttfamily\footnotesize, frame=single]
\end{lstlisting}


\begin{center}
\end{center}


\end{document}